\documentclass[paper=a4,fontsize=11pt]{scrartcl}
\usepackage{underscore,changepage,booktabs,xcolor}
\usepackage[T1]{fontenc}
\usepackage[english]{babel} % English language/hyphenation
\usepackage[protrusion=true,expansion=true]{microtype}
\usepackage{amsmath,amsfonts,amsthm,url,xspace}
\usepackage[pdftex]{graphicx}
\newcommand{\eg}{e.g.,\xspace}
\newcommand{\bigeg}{E.g.,\xspace}
\newcommand{\etal}{\textit{et~al.\xspace}}
\newcommand{\etc}{etc.\@\xspace}
\newcommand{\ie}{i.e.,\xspace}
\newcommand{\bigie}{I.e.,\xspace}
\newcommand{\secref}[1]{section~\ref{#1}}
\newcommand{\Secref}[1]{Section~\ref{#1}}
%%% Custom sectioning
\usepackage{sectsty}
\allsectionsfont{\centering\normalfont\scshape}
%%% Custom headers/footers (fancyhdr package)
\usepackage{fancyhdr}
\pagestyle{fancyplain}
\fancyhead{}                        % No page header
\fancyfoot[L]{}                     % Empty
\fancyfoot[C]{}                     % Empty
\fancyfoot[R]{\thepage}             % Pagenumbering
\renewcommand{\headrulewidth}{0pt}  % Remove header underlines
\renewcommand{\footrulewidth}{0pt}  % Remove footer underlines
\setlength{\headheight}{13.6pt}
%%% Equation and float numbering
\numberwithin{equation}{section}    % Equationnumbering: section.eq#
\numberwithin{figure}{section}      % Figurenumbering: section.fig#
\numberwithin{table}{section}       % Tablenumbering: section.tab#
%%% Institution and Authors
\newcommand{\horrule}[1]{\rule{\linewidth}{#1}}     % Horizontal rule
\title{
  % \vspace{-1in}
  \usefont{OT1}{bch}{b}{n}
  \normalfont\normalsize\textsc{HLTCOE, Johns Hopkins University}\\[25pt]
  \horrule{0.5pt}\\[0.4cm]
  \huge Vertex Nomination on the Cold Start Knowledge Graph\\
  \horrule{2pt}\\[0.5cm]
}
\author{
  \normalfont\normalsize
  Pushpendre Rastogi\\
  %[-3pt]\normalsize\today%%Optional Date
}
\date{}

% -------------- %
% Talking Points %
% -------------- %
% Surdenaue says that there are 41 relations. but there doesn't seem
% to be 41 relations.
\newcommand{\newrel}{41} 
\renewcommand{\cite}[1]{\textcolor{red}{#1}}
\renewcommand{\r}[1]{\textcolor{red}{#1}}
\begin{document}
\maketitle
\section{Introduction}
\label{sec:introduction}
The TAC Cold Start Knowledge Base Population (CS-KBP) shared task is organized by
NIST to evaluate systems that can build a knowledge base (KB) from raw natural
language text~\cite{TAC-KBP-DESC}. 
At the beginning of the task the participants
are provided a KB schema and a set of documents 
from which they have to automatically extract entities and their relations and
populate a KB. The KB should be an accurate representation of the information 
in the text documents. The competing systems 
extract three types of entities: \textsc{PER, ORG, LOC}
\footnote{PER, ORG and LOC are abbreviations of Person, Organization and
  Location respectively.}
and a larger number of relations between those entities.\footnote{
Relations are also called slots in prior work. In CS-KBP 2015 task
the number of regular relations was 41. Out of these 41 relations, 
26 were entity-valued, that is their arguments were themselves entities.
Each slot has atleast one inverse slot, which may or may not be a pre-existing
slot. For example \texttt{org:shareholders} has three inverse slots
\texttt{\{per,org,gpe\}:holds_shares_in}. All \texttt{gpe:*} slots are inverse
slots that did not exist previously. The total
number of inverse slot that did not previously exist is 22. So the total
number of slots is 63.}

BBN participated in the CS-KBP task in 2015 and it was the top performing
participant in terms of precision and overall F1 according to the official 
results as of November 2015~\cite{BBN-System}. In order to assess the 
feasibility of performing \textsc{vertex nomination} on KBs built with a
fixed ontology of a small number of relations we accessed BBN's \texttt{BBN2} 
submission to the CS-KBP task and performed vertex nomination experiments on 
the \texttt{BBN2} database. \Secref{sec:data} describes the \texttt{BBN2} KB.


\section{Data}
\label{sec:data}

\subsection{Data Cleaning}
\label{sec:data-cleaning}

\section{Experiments}
\label{sec:experiments}


\subsection{Query Types}
\label{sec:query-types}

\subsection{Evaluation Metrics}
\label{sec:evaluation-metrics}

\subsection{Unary Heuristic}
\label{sec:unary_heuristic}

\subsection{Using Matrix Factorization}
\label{sec:using-matr-fact}

\subsection{VN as KBC}
\label{sec:vn-as-kbc}

\subsection{VN through MAD}
\label{sec:vn-through-mad}

\subsection{VN with Semi-Supervised Information}
\label{sec:vn-with-semi}


\bibliographystyle{plain}
\bibliography{references.bib}
\end{document}