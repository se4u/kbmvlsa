\documentclass{tufte-handout}
% -------- %
% Packages %
% -------- %
% \usepackage[T1]{fontenc}
\usepackage{graphicx,amsmath,amssymb,url,xspace,booktabs,xcolor}
\usepackage[]{hyperref}
\hypersetup{colorlinks,%
linkcolor={red!50!black},%
citecolor={blue!50!black},%
urlcolor={blue!80!black}%
}
% tex.stackexchange.com/questions/2291/
% how-do-i-change-the-enumerate-list-format-to-use-letters-instead-of-the-defaul#comment3172_2294
\usepackage[shortlabels]{enumitem}
% tex.stackexchange.com/questions/171803/
% change-font-size-of-the-verbatim-environment
\usepackage{fancyvrb}
\usepackage{microtype}
\usepackage[acronym]{glossaries}
\usepackage[]{todonotes} % insert [disable] to disable all notes.
\usepackage{array}

% http://tex.stackexchange.com/questions/121601
% automatically-wrap-the-text-in-verbatim
\usepackage{listings}
\lstset{basicstyle=\small\ttfamily,%
  columns=flexible,%
  breaklines=true,%
  linewidth=17cm,%
  xleftmargin=-1cm,%
  xrightmargin=-1cm}
\newcolumntype{C}{>{$}c<{$}}
\newcolumntype{L}{>{$}l<{$}}
\newcolumntype{R}{>{$}r<{$}}
% -------- %
% Commands %
% -------- %
% Color Me Red
\newcommand{\cmr}[1]{{\color{red} #1}}
\newcommand{\note}[1]{\todo[author=Pushpendre,color=blue!40,size=\small,fancyline,inline]{#1}}
\newcommand{\Todo}[1]{\todo[author=Pushpendre,size=\small,inline]{#1}}
\newcommand{\eg}{e.g.,\xspace}
\newcommand{\bigeg}{E.g.,\xspace}
\newcommand{\etal}{\textit{et~al.\xspace}}
\newcommand{\etc}{etc.\@\xspace}
\newcommand{\ie}{i.e.,\xspace}
\newcommand{\bigie}{I.e.,\xspace}
\renewcommand{\cite}[1]{\textcolor{red}{#1}}
\newcommand{\alert}[1]{\textcolor{red}{#1}}
\newcommand{\remove}[1]{} % Change to {\remove}[0]{} to bring back
\newcommand{\zset}{\left\{ 0 \right\}}
\newcommand{\figref}[1]{Figure~\ref{#1}}
\newcommand{\tabref}[1]{Table~\ref{#1}}
\newcommand{\thref}[1]{Theorem~\ref{#1}}
\newcommand{\lemref}[1]{Lemma~\ref{#1}}
\newcommand{\secref}[1]{Section~\ref{#1}}

\title{2 Models for Vertex Nomination--Recommendations--Single Instance Multi
  Labeling in bipartite graphs.}
\author{PR, VL, BVD}
% ----------------------- %
% Document Class Settings %
% ----------------------- %
\IfFileExists{bergamo.sty}{\usepackage[osf]{bergamo}}{}% Bembo
\IfFileExists{chantill.sty}{\usepackage{chantill}}{}% Gill Sans
\setcaptionfont{%
\color{blue}% <-- set color here
}
% --------------------------------------- %
% Reset the table and figure environments %
% --------------------------------------- %
\makeatletter
\renewenvironment{figure}[1][htbp]{%
\@tufte@orig@float{figure}[#1]}{%
\@tufte@orig@endfloat}%
\renewenvironment{table}[1][htbp]{%
\@tufte@orig@float{table}[#1]}{%
\@tufte@orig@endfloat}%
\makeatother

\begin{document}
\maketitle
\section{Introduction}
\label{sec:introduction}
Cold start Knowledge base population is an existing/important information extraction task. 

The information that is most commonly expressed in business/intelligence type documents is 
information about people, locations and organizations. Specifically things are place of birth,
nationality, employer, ethnicity, titles, occupations, business lines etc. are important. 

Considerable progress has been made in natural language processing of clean/grammatical text.
In fact automatically generated paleontology knowledge
bases made by parsing research papers now have the same precision as human created ones on some measures\cite{paleodeepdive}.
OTOH, automatica information extraction from noisy web text is still much more imprecise than human performance.
The performance is low enough to warrant annual NIST evaluations and they have consistenly found for the last two years
that the performance is at 50\% of humans\cite{surdenaue-tackbp-overviews}. 
\todo{mihael's TACKBP overview of 2015 is not out yet}. However the performance of relation extractors varies by a lot.
For example extracting the \textbf{roles} (or occupations) of a person from appositives is a fairly 
high precision task.

In this paper we address the following future-looking question: Once we have extracted certain 
labels for people, locations and entities from a text collection what inferences can we draw from 
those facts? As a concrete example, let's say that we extract the type of information that is present 
in wikipedia infoboxes about people, locations and organizations, such as their date of births, their 
nationality, their employers and their role in society, or their occupation. Now assume that 
an analyst selects some of people as \textit{people of interest}. How should we rank/nominate/recommend 
other people as people of interest?

This paper presents two models of such data and experimentally compares them to previous related work.

\newthought{Model 1} Embeddings based models that represent labels. See \figref{fig:tmp}
\newthought{Model 2} Vertex nomination \cite{lyzinski} specialized to bipartite
graphs.

\begin{figure}[htbp]
  \centering
  \includegraphics[width=1.7\linewidth]{tmp.jpg}
  \caption{Temporary Figure to describe Two possible models for vertex
    nomination in bipartite graphs.}
  \label{fig:tmp}
\end{figure}

\clearpage
\section{Extensions}
\label{sec:extensions}
Ascertaining whether two people are likely to be friends based on their common labels.

\bibliographystyle{plain}
\bibliography{references}
\end{document}